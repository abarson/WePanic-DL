\documentclass{article}
\usepackage{fancyhdr}
\usepackage[margin=1in]{geometry}
\usepackage{amsmath}
\usepackage{amsfonts}
\usepackage{graphicx}

\pagestyle{fancy}
\lhead{CS201 (Operating Systems): Setting up easy SSH connections}
\rhead{UVM, Fall 2018 \\ Daniel Berenberg}

\begin{document}
\subsubsection*{Introduction}
This document is meant to show a quick tutorial to streamlining your ssh connections.
By the end of this document you will have enabled yourself to enter commands of this flavor:
\begin{gather*}
\mathrm{\texttt{ssh} \ \textit{my-remote-server}} \\
\mathrm{\texttt{scp} \ \textit{my-remote-server} \ \texttt{local\_file}:\texttt{remote\_destination}}
\end{gather*}
\textit{without} needing to enter your password (usually) or enter a complicated hostname. This is a method called \textbf{ssh aliasing}.
%--------------------------------------------
\subsubsection*{Setup (assuming a linux-like machine)}
\begin{enumerate}
\item Enter your \texttt{\$HOME/.ssh} folder: \texttt{cd \$HOME/.ssh}. 
\item Make an ssh key pair:
    \begin{quote}
        \texttt{ssh-keygen -t rsa}
    \end{quote}
    Press \texttt{enter} through the prompted messages; you DO NOT want an alternate location nor a passphrase.

\item Next we are going to create a file called \texttt{config} in our \texttt{\$HOME/.ssh} folder. Below is a screenshot of my \texttt{config}
      file. \\
        \includegraphics{}
%\item Create a file called \texttt{config} and enter the following information:
%    \begin{quote}
%        \texttt{Host * \\
%        IdentityFile ~/.ssh/id\_rsa
%        }
%    \end{quote}
%    In the same file, enter the your custom remote host information:
%    \begin{quote}
%        \texttt{Host \textit{custom-host-alias-name}} (e.g. \texttt{kaladin})\\
%        \texttt{User \textit{your-username-on-this-host}} (in this case, your netid) \\ 
%        \texttt{Hostname} \texttt{the\_actual\_hostname}
%    \end{quote}
%    In the case of WePanic, the only custom entry in this part of the file is the \texttt{Host} parameter. The
%    Username is \texttt{bloodletter} and the Hostname is \texttt{wepanic-dl.eastus.cloudapp.azure.com}. \\
%    You have now aliased your ssh. Trying \texttt{ssh} \textit{custom-host-alias} will work, but still prompt you
%    with a password. We'll fix that in the next step.


\item Install the ssh key on the remote machine.
    \begin{quote}
        \texttt{cat ~/.ssh/\textbf{id\_rsa.pub} | ssh user@remote "mkdir -p ~/.ssh; cat >> ~/.ssh/authorized\_keys"}
    \end{quote}
        Make sure to use \texttt{>>} not \texttt{>} to \textit{append} to the file rather than overwriting it! 
\end{enumerate}
\end{document}
