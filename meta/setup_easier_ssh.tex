\documentclass{article}
\usepackage{fancyhdr}
\usepackage[margin=1in]{geometry}
\usepackage{amsmath}
\usepackage{amsfonts}

\pagestyle{fancy}
\lhead{WePanic-DL: Setting up easy SSH connections}
\rhead{UVM, Summer 2018 \\ Daniel Berenberg}

\begin{document}
\subsubsection*{Introduction}
This document is meant to show a quick tutorial to streamlining your ssh connections.
By the end of this document you will have enabled the ability to enter commands of this flavor:
\begin{gather*}
\mathrm{\texttt{ssh} \ \textit{my-remote-server}} \\
\mathrm{\texttt{scp} \ \textit{my-remote-server} \ \texttt{local\_file}:\texttt{remote\_destination}}
\end{gather*}
\textit{without} needing to enter your password. This is a method called \textbf{ssh aliasing}.
Anyway, without further adieu, let's set it up. \\ \\
%--------------------------------------------
\subsubsection*{Setup}
\begin{enumerate}
\item Enter your \texttt{~/.ssh} folder.
\item Create a file called \texttt{config} and enter the following information (verbatim):
    \begin{quote}
        \texttt{Host * \\
        AddKeysToAgent yes \\
        UseKeychain yes \\
        IdentityFile ~/.ssh/id\_rsa \\
        }
    \end{quote}
    In the same file, enter the your custom remote host information:
    \begin{quote}
        \texttt{Host} \textit{custom-host-alias} \\
        \texttt{User} \textit{your-username-on-this-host} \\ 
        \texttt{Hostname} \texttt{the\_actual\_hostname}
    \end{quote}
    In the case of WePanic, the only custom entry in this part of the file is the \texttt{Host} parameter. The
    Username is \texttt{bloodletter} and the Hostname is \texttt{wepanic-dl.eastus.cloudapp.azure.com}. \\
    You have now aliased your ssh. Trying \texttt{ssh} \textit{custom-host-alias} will work, but still prompt you
    with a password. We'll fix that in the next step.

\item Make an ssh key pair:
    \begin{quote}
        \texttt{ssh-keygen -t rsa}
    \end{quote}
    Press \texttt{enter} through the prompted messages; you DO NOT want an alternate location nor a passphrase.

\item Install the ssh key on the remote machine.
    \begin{quote}
        \texttt{cat ~/.ssh/\textbf{id\_rsa.pub} | ssh user@remote "mkdir -p ~/.ssh; cat >> ~/.ssh/authorized\_keys"}
    \end{quote}
        Make sure to use \texttt{>>} not \texttt{>} to \textit{append} to the file rather than overwriting it! 
\end{enumerate}
\end{document}
